%%%%%%%%%%%%%%%%%%%%%%%%%%%%%%%%%%%%%%%%%
% Short Sectioned Assignment LaTeX Template Version 1.0 (5/5/12)
% This template has been downloaded from: http://www.LaTeXTemplates.com
% Original author:  Frits Wenneker (http://www.howtotex.com)
% License: CC BY-NC-SA 3.0 (http://creativecommons.org/licenses/by-nc-sa/3.0/)
%%%%%%%%%%%%%%%%%%%%%%%%%%%%%%%%%%%%%%%%%

%----------------------------------------------------------------------------------------
%	PACKAGES AND OTHER DOCUMENT CONFIGURATIONS
%----------------------------------------------------------------------------------------

\documentclass[paper=a4, fontsize=11pt]{scrartcl} % A4 paper and 11pt font size

% ---- Entrada y salida de texto -----

\usepackage[T1]{fontenc} % Use 8-bit encoding that has 256 glyphs
\usepackage[utf8]{inputenc}

% ---- Idioma --------

\usepackage[spanish, es-tabla]{babel} % Selecciona el español para palabras introducidas automáticamente, p.ej. "septiembre" en la fecha y especifica que se use la palabra Tabla en vez de Cuadro

% ---- Otros paquetes ----

\usepackage{amsmath,amsfonts,amsthm} % Math packages
\usepackage{graphics,graphicx, floatrow} %para incluir imágenes y notas en las imágenes
\usepackage{graphics,graphicx, float} %para incluir imágenes y colocarlas
\usepackage{hyperref} % url in references
\usepackage{textcomp}
\usepackage{listings}
\usepackage{titlesec}

% Para hacer tablas comlejas
\usepackage{multirow}
\usepackage{threeparttable}

\usepackage{fancyhdr} % Custom headers and footers
\pagestyle{fancyplain} % Makes all pages in the document conform to the custom headers and footers
\fancyhead{} % No page header - if you want one, create it in the same way as the footers below
\fancyfoot[L]{} % Empty left footer
\fancyfoot[C]{} % Empty center footer
\fancyfoot[R]{\thepage} % Page numbering for right footer
\renewcommand{\headrulewidth}{0pt} % Remove header underlines
\renewcommand{\footrulewidth}{0pt} % Remove footer underlines
\setlength{\headheight}{13.6pt} % Customize the height of the header

\numberwithin{equation}{section} % Number equations within sections (i.e. 1.1, 1.2, 2.1, 2.2 instead of 1, 2, 3, 4)
\numberwithin{figure}{section} % Number figures within sections (i.e. 1.1, 1.2, 2.1, 2.2 instead of 1, 2, 3, 4)
\numberwithin{table}{section} % Number tables within sections (i.e. 1.1, 1.2, 2.1, 2.2 instead of 1, 2, 3, 4)

\setlength\parindent{0pt} % Removes all indentation from paragraphs - comment this line for an assignment with lots of text

\newcommand{\horrule}[1]{\rule{\linewidth}{#1}} % Create horizontal rule command with 1 argument of height

\usepackage{textcomp}
\usepackage{listings}

%----------------------------------------------------------------------------------------
%	DATOS
%----------------------------------------------------------------------------------------

\newcommand{\myName}{Francisco Javier Bolívar Lupiáñez}
\newcommand{\myDegree}{Grado en Ingeniería Informática}
\newcommand{\myFaculty}{E. T. S. de Ingenierías Informática y de Telecomunicación}
\newcommand{\myDepartment}{Lenguajes y Sistemas de Información}
\newcommand{\myUniversity}{\protect{Universidad de Granada}}
\newcommand{\myLocation}{Granada}
\newcommand{\myTime}{\today}
\newcommand{\myTitle}{Práctica 1}
\newcommand{\mySubtitle}{Implementación distribuida de un algoritmo paralelo de datos usando MPI}
\newcommand{\mySubject}{Programación Paralela}
\newcommand{\myYear}{2015-2016}

%----------------------------------------------------------------------------------------
%	PORTADA
%----------------------------------------------------------------------------------------


\title{	
	\normalfont \normalsize 
	\textsc{{\bf \mySubject \space (\myYear)} \\ \myDepartment} \\[20pt] % Your university, school and/or department name(s)
	\textsc{\myDegree \\[10pt] \myFaculty \\ \myUniversity} \\[25pt]
	\horrule{0.5pt} \\[0.4cm] % Thin top horizontal rule
	\huge \myTitle: \mySubtitle \\ % The assignment title
	\horrule{2pt} \\[0.5cm] % Thick bottom horizontal rule
	\normalfont \normalsize
}

\author{\myName} % Nombre y apellidos

\date{\myTime} % Incluye la fecha actual
%----------------------------------------------------------------------------------------
%	INDICE
%----------------------------------------------------------------------------------------

\begin{document}

\lstdefinestyle{C} {
	basicstyle=\scriptsize,
	frame=single,
	language=C,
	numbers=left
}
	
\setcounter{page}{0}

\maketitle % Muestra el Título
\thispagestyle{empty}

\newpage %inserta un salto de página

\tableofcontents % para generar el índice de contenidos

\listoffigures

\newpage

%----------------------------------------------------------------------------------------
%	DOCUMENTO
%----------------------------------------------------------------------------------------

\section{Planteamiento}

En esta práctica se llevará a cabo la paralelización del algoritmo Floyd para la búsqueda del camino más corto en un grafo.

Se desarrollarán dos versiones:
\begin{itemize}
	\item \textbf{Unidimensional}: Se repartirán las filas de la matriz a los procesos.
	\item \textbf{Bidimensional}: Se repartirán submatrices de la matriz a los procesos.
\end{itemize}

\subsection{Algoritmo de Floyd}

El algoritmo de Floyd deriva una matriz en N pasos (tantos como número de nodos), obteniendo en cada paso una matriz intermedia con el camino más corto entre cada par de nodos.

\subsubsection{Pseudocódigo}

\begin{lstlisting}[style=c]
I_ij = A;
for k = 0 to N-1
	for i = 0 to N-1
		for j = 0 to N-1
			I_ij = min(I_ij, I_ik + I_kj);
\end{lstlisting}

\section{Solución}

\subsection{Versión unidimensional}

Para solucionarlo con este enfoque, asumiendo que el tamaño del problema es divisible entre el número de procesos, cada proceso tendrá una matriz local de tamaño $N/P \times N$.
\\ \\
El reparto se realizará por bloques. Es decir. Al primer proceso le corresponderán las primeras N/P filas, al segundo las siguientes... Por ejemplo, si el tamaño del problema es 8 y tenemos 4 procesos, al $P_{0}$ le corresponderás las filas 0 y 1, al $P_{1}$ las 2 y 3, al $P_{2}$ las 4 y 5 y al $P_{3}$ las 6 y 7.
\\ \\
En el cálculo de cada submatriz resultado, cada proceso necesitará, en el paso k, la fila k y puede tener suerte y ser suya o no y corresponderle a otro proceso. Entonces debería comunicarse con éste para poder realizar el cálculo. Por tanto, en cada iteración del primer bucle k, el proceso detectará si la fila k le pertenece y si es así, hace un \textit{broadcast} al resto de procesos.
\\ \\
Por tanto, para solucionar el problema, nos basta con un \textit{scatter} para repartir la matriz, un \textit{broadcast} para difundir cada fila k y un \textit{gather} para recolectar la matriz resultado y el único problema que nos podríamos encontrar es el traducir de local a global un índice según lo que se necesite.

\end{document}